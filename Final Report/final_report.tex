\documentclass[english]{article}\usepackage[]{graphicx}\usepackage[]{color}
\usepackage{alltt}
\usepackage[T1]{fontenc}
\usepackage[latin9]{inputenc}
\usepackage{geometry}
\geometry{verbose}
\setcounter{secnumdepth}{2}
\setcounter{tocdepth}{2}
\usepackage{amsmath}
\usepackage{graphicx}
\usepackage{esint}
\usepackage{babel}
\IfFileExists{upquote.sty}{\usepackage{upquote}}{}
\begin{document}

\title{Autonomous Robots and Environmental Mapping}

\author{Mitchell Horning, Matthew Lin, Siddarth Srinivasan, Simon Zou}

\maketitle

\tableofcontents

\section{Introduction}

In the field of signal processing, compressed sensing is a technique used for 
reconstructing signals and images from very few samples and is commonly used in 
fMRIs and medical imaging. We aim to apply compressed sensing to allow 
autonomous robots to map an environment and identify areas of interest which 
are significantly different from the surrounding area. In our experiment, we  
program a robot equipped with sensors that travels along a path and will do on-
board summation of sensor readings and send that sum (or path integral) to a 
server. We then apply compressed sensing to the the data based on the readings 
and the paths the robot took and attempt to reconstruct the environment the 
robot traveled over from the data. 


\section{Experiment settings}

\section{Models and assumptions}

\section{Algorithm for solving the inverse problem}

\section{Conclusions and Further Work}

\end{document}
