\documentclass[english]{article}\usepackage[]{graphicx}\usepackage[]{color}
\usepackage{alltt}
\usepackage[T1]{fontenc}
\usepackage[latin9]{inputenc}
\usepackage{geometry}
\geometry{verbose}
\setcounter{secnumdepth}{2}
\setcounter{tocdepth}{2}
\usepackage{amsmath}
\usepackage{graphicx}
\usepackage{esint}
\usepackage{babel}
\usepackage{verbatim} % for \begin{comment}
\IfFileExists{upquote.sty}{\usepackage{upquote}}{}
\begin{document}

\title{Autonomous Robots and Environmental Mapping}

\author{Mitchell Horning, Matthew Lin, Siddarth Srinivasan, Simon Zou}

\maketitle

\begin{abstract}

In the field of signal processing, compressed sensing is a technique used for 
reconstructing signals and images from very few samples and is commonly used in 
fMRIs and medical imaging. We aim to apply compressed sensing to allow 
autonomous robots to map an environment and identify areas of interest which 
are significantly different from the surrounding area. In our experiment, we 
program a robot equipped with sensors to travel along a path, do on-board 
summation of sensor readings, and send that sum (or path integral) to a 
server. We then apply reconstruction algorithms to the data, which consists of sensor readings 
and the travelled paths, to reconstruct the environment.

\end{abstract}

\tableofcontents

\section{Introduction}

\begin{comment}
Discuss Compressed Sensing

\end{comment}

\section{Experiment settings}
\subsection{Testbed}
\subsection{Vehicle hardware}
\subsection{Server}
\begin{comment}
Description of testbed, hardware, software, logic
Sid's flowchart
\end{comment}

\section{Models and assumptions}
\subsection{Constraints}
\begin{comment}
Description of constraints for our problem
limited bandwidth, data storage
Also assumptions about simple piecewise environments
\end{comment}

\section{Algorithm for solving the inverse problem}
\begin{comment}
\end{comment}
\section{Conclusions and Further Work}

\end{document}
