\documentclass[english]{article}\usepackage[]{graphicx}\usepackage[]{color}
\usepackage{alltt}
\usepackage[T1]{fontenc}
\usepackage[latin9]{inputenc}
\usepackage{geometry}
\geometry{verbose}
\setcounter{secnumdepth}{2}
\setcounter{tocdepth}{2}
\usepackage{amsmath}
\usepackage{graphicx}
\usepackage{esint}
\usepackage{babel}
\usepackage{verbatim} % for \begin{comment}
\IfFileExists{upquote.sty}{\usepackage{upquote}}{}
\begin{document}

\title{Autonomous Robots and Environmental Mapping}

\author{Mitchell Horning, Matthew Lin, Siddarth Srinivasan, Simon Zou}

\maketitle

\begin{abstract}

In the field of signal processing, compressed sensing is a technique used for 
reconstructing sparse signals and images from very few samples and is commonly 
used in fMRIs and medical imaging. We aim to use this technique to allow 
autonomous robots to map an environment and identify areas of interest that 
are significantly different from the surrounding area. In our experiment, we 
programmed a robot equipped with a reflectance sensor to travel along predetermined
straight-line paths, perform on-board summations of sensor readings along that path, and send that sum to a remote server after each path. An overhead video camera is used to track
the robot's position as it travels on the test bed. Each data point consists of
the robot's start position, end position and the sum of sensor readings along
that path. Once all the data has been collected, we applied reconstruction algorithms to the data, to reconstruct the environment. When using an adaptive pathing scheme, preliminary simulations of our experiment show faithful reconstruction of 100x100 images with only 100 data points (err. = 0.28). and our hardware is prepared to begin 
performing the proposed task.

\end{abstract}

\tableofcontents

\section{Introduction}

\begin{comment}
Discuss Compressed Sensing

\end{comment}

Compressed sensing describes a technique in which a signal is reconstructed by 
solving an underdetermined linear system. Doing so allows a signal to be 
found from a relatively small amount of data.

\section{Experiment settings}
\subsection{Testbed}
\subsection{Vehicle hardware}
\subsection{Server}
\begin{comment}
Description of testbed, hardware, software, logic
Sid's flowchart
\end{comment}

\section{Models and assumptions}
\subsection{Constraints}
\begin{comment}
Description of constraints for our problem
limited bandwidth, data storage
Also assumptions about simple piecewise environments
\end{comment}

\section{Algorithm for solving the inverse problem}
\begin{comment}
\end{comment}
\section{Conclusions and Further Work}

\end{document}
