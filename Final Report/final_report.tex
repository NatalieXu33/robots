\documentclass[english]{article}\usepackage[]{graphicx}\usepackage[]{color}
\usepackage{alltt}
\usepackage[T1]{fontenc}
\usepackage[latin9]{inputenc}
\usepackage{geometry}
\geometry{verbose}
\setcounter{secnumdepth}{2}
\setcounter{tocdepth}{2}
\usepackage{amsmath}
\usepackage{graphicx}
\usepackage{esint}
\usepackage{babel}
\IfFileExists{upquote.sty}{\usepackage{upquote}}{}
\begin{document}

\title{Problem description and current progress}


\author{Mitchell Horning, Matthew Lin, Siddarth Srinivasan, Simon Zou}

\maketitle

\section{Introduction}

Mapping environment variables by robots has drawn a lot of interests
these days. In particular, we want to identify the regions of interest,
where the values of certain environment variable is significantly
different from surrounging area, though almost uniform inside each
region. Normally the programmed robots are sent out in the field,
moving along designated paths and collecting the measurements of the
environment variable on the way. Each robot is equipped with a positioning
device and are able to get measurements at a given frequency. It can
also send stored data back to the base by wireless communication.
If the coordinates at every point on the path can be recorded and
the robot can send measurements back in real time, then theoretically
the values of the environment variable at every point can be obtained.
However, it is impractical due to limited bandwith. To address the
bandwith limitation, we propose a different strategy. The robot is
programmed to perform on-board summation of the measurements obtained
along each path (so-called path integral), and only sending the path
integral to the base. Ideally the paths are line segments, so only
the coordinates of begin and end of each line segment are needed in
order to determine the paths. These path integrals of measurements
along with the path information are post-processed using the reconstruction
techniques of computed tomography and compressed sensing. Hopefully
we are able to reconstruct the environment variables in the field.


\section{Experiment settings}

To be added.


\section{Models and assumptions}

Without loss of generality, we assume the area to be explored is a
rectangle denoted by $\Omega$, and the interested environment variable
$u(x,y)$ is a piece-wise constant function defined on $\Omega$.
See Figure \ref{fig:An-illustration-of-paths} for an illustration.
Assume the robot travels through $n$ different paths, which are denoted
by $C_{1},\ldots,C_{n}$. Along each path $C_{k}$, the integral of
environment variable $u$ is obtained, which is denoted by $g_{k}$.
The path integrals are written as
\begin{equation}
g_{k}=\int_{C_{k}}u(x,y)\mathrm{\,{d}\Gamma},\; k=1,\ldots,n.\label{eq:path-integral}
\end{equation}


\begin{figure}[h]
\begin{centering}
\par\end{centering}

\caption{\label{fig:An-illustration-of-paths}An illustration of the paths
of robots. The shaded regions denote the area of interest (unknown),
where the values of environment variable $u$ is significantly different
from the surroundings. The size of the unit voxel (each small rectangle)
is determined by the accuracy of the positioning system.}
\end{figure}


For computational purpose, the whole domain is discretized into rectangular
voxels (See Figure \ref{fig:An-illustration-of-paths}). Each path
$C_{k}$ defines a weight $a_{k,ij}$ for each voxel $(i,j)$. If
$C_{k}$ does not intersect with voxel $(i,j)$, then $a_{k,ij}=0$,
otherwise $a_{k,ij}$ is defined to be proportional to the length
of the part of $C_{k}$ that falls within voxel $(i,j)$. It is also
assumed that the value of $u$ is a constant within each voxel $(i,j)$,
which is denoted by $u_{ij}$. After discretization, Eq (\ref{eq:path-integral})
becomes

\begin{equation}
g_{k}=\sum_{i}\sum_{j}a_{k,ij}u_{ij}.\label{eq:path-integral-discrete}
\end{equation}
Let $g=(g_{1},\ldots,g_{n})^{t}$, $u=(u_{ij})$, then $g$ and $u$
have a linear relation, which is written as 
\begin{equation}
g=Au,\label{eq:system-eq}
\end{equation}
where the linear operator $A$ is specified in Eq (\ref{eq:path-integral-discrete}).
Eq (\ref{eq:system-eq}) is the model equation, which poses an inverse
problem. In this equation, $g$ can be obtained from the experiment,
$A$ is determined by user-specified paths for the robot, which can
be calculated offline. $u$ is the variable that we want to solve.
Of course, if the paths form a complete raster scan over the whole
domain, then theoretically Eq (\ref{eq:system-eq}) has a unique solution.
For economical reasons, it is desirable to use much fewer paths and
still being able to reconstruct a solution for $u$. Inspired by compresed
sensing based image reconstruction techniques, it is hopeful that
reconstruction for our under-determined system can be done.


\section{Algorithm for solving the inverse problem}

Based on the observation that the image $u$ to be reconstructed is
piecewise constant, it's gradient $\nabla u$ is sparse over
the domain. Inspired by the compressed sensing theory, we postulate
that the actual solution $u$ should minimize the energy functional 

\begin{equation}
J(u)=\frac{\mu}{2}\|Au-g\|_{2}^{2}+\alpha|u|_{1}+\beta(|\nabla_{x}u|_{1}+|\nabla_{y}u|_{1}),\label{eq:energy-functional}
\end{equation}
where $\alpha,\beta$ are some constants to be tuned depending on
applications. Here we use Split-Bregman method to solve the problem
$\underset{u}{{\text{{min}}}}J(u)$. More exactly, we rewrite minimization
of (\ref{eq:energy-functional}) as a constraint problem 
\[
\underset{u,d,d_{x},d_{y}}{{\text{{min}}}}\frac{\mu}{2}\|Au-g\|_{2}^{2}+\alpha|d|_{1}+\beta(|d_{x}|_{1}+|d_{y}|_{1})\quad\text{such that}\; d=u,\, d_{x}=\nabla_{x}u,\, d_{y}=\nabla_{y}u.
\]
And its associated Augmented Lagrangian functional $L(u,\, d,\, d_{x,}\, d_{y},\, b,\, b_{x},\, b_{y})$ is 
\[
\frac{\mu}{2}\|Au-g\|_{2}^{2}+\alpha|d|_{1}+\beta(|d_{x}|_{1}+|d_{y}|_{1})+\frac{1}{2}\lambda_{1}\|d-u+b\|_{2}^{2}+\frac{{1}}{2}\lambda_{2}(\|d_{x}-\nabla_{x}u+b_{x}\|_{2}^{2}+\|d_{y}-\nabla_{y}u+b_{y}\|_{2}^{2}).
\]
 The solution process is iterative, with each iteration consisting
of three major steps:
\begin{enumerate}
\item Update $u$ by solving the Least-Squares problem
\begin{equation}
\underset{u}{{\text{{min}}}}\frac{\mu}{2}\|Au-g\|_{2}^{2}+\frac{1}{2}\lambda_{1}\|d-u+b\|_{2}^{2}+\frac{{1}}{2}\lambda_{2}(\|d_{x}-\nabla_{x}u+b_{x}\|_{2}^{2}+\|d_{y}-\nabla_{y}u+b_{y}\|_{2}^{2}).\label{eq:LS-problem}
\end{equation}

\item Update $d,\, d_{x},\, d_{y}$ by soft-thresholding (shrinkage)
\[
d=\text{{sign}}(u-b)\cdot\text{{max}}(|u-b|-\frac{\alpha}{\lambda_{1}},0).
\]
\[
d_{x}=\text{{sign}}(\nabla_{x}u-b_{x})\cdot\text{{max}}(|\nabla_{x}u-b_{x}|-\frac{\beta}{\lambda_{2}},0).
\]
\[
d_{y}=\text{{sign}}(\nabla_{y}u-b_{y})\cdot\text{{max}}(|\nabla_{y}u-b_{y}|-\frac{\beta}{\lambda_{2}},0).
\]
\[
\]

\item Update auxillary variables $b,\, b_{x},\, b_{y}$
\[
b=d-u+b.
\]
\[
b_{x}=d_{x}-\nabla_{x}u+b_{x}.
\]
\[
b_{y}=d_{y}-\nabla_{y}u+b_{y}.
\]

\end{enumerate}
In step 1, we need to solve following linear equation
\begin{equation}
(\mu A^{T}A+\lambda_{1}I+\lambda_{2}\nabla_{x}^{T}\nabla_{x}+\lambda_{2}\nabla_{y}^{T}\nabla_{y})u=\mu A^{T}g+\lambda_{1}(d+b)+\lambda_{2}\nabla_{x}^{T}(d_{x}+b_{x})+\lambda_{2}\nabla_{y}^{T}(d_{y}+b_{y}).\label{eq:Laplace-equation}
\end{equation}
Since it is a positive-definite system, it can be solved by Conjugate
Gradient method, with the preconditioner give by the linear operator
\begin{equation}
(\lambda_{1}I+\lambda_{2}\nabla_{x}^{T}\nabla_{x}+\lambda_{2}\nabla_{y}^{T}\nabla_{y})^{-1}.\label{eq:inverse_Laplace}
\end{equation}
Because $\nabla_{x},\,\nabla_{y}$ are both circulant matrices using
finite-difference approximation assuming periodic boundary condition,
(\ref{eq:inverse_Laplace}) can be solved by Fourier transform and
its inverse. 
\[
\]
\[
\]

\end{document}
